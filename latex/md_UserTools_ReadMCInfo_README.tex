This Tool appears to have no dependancies on upstream tools. It populates the event\+Primaries, event\+Secondaries, and event\+True\+Captures members of the \mbox{\hyperlink{classDataModel}{Data\+Model}}, by reading either the MCVERTEX, MCVECT and SCNDPRT and SCNDPRTVC data banks for zbs files, or the SECONDARY branch\textquotesingle{}s atmpd-\/style arrays for SKROOT files -\/ Note these arrays are only populated when generating files with SKG4, and explicitly given the XXX flag. (skdetsim does not produce this branch). The \mbox{\hyperlink{classWriteOutput}{Write\+Output}} Tool will then put these into the \textquotesingle{}mc\textquotesingle{} TTree in the output file, although the \mbox{\hyperlink{classWriteOutput}{Write\+Output}} tool also writes out other output, so has other Tool dependencies.

T0 is obtained by\+: {\ttfamily trginfo\+\_\+(\&\+T0)} and used for all primary particle times, and added to all secondary particle times.

primaries are retrieved by calling {\ttfamily skgetv\+\_\+();} (\$\+SKOFL\+\_\+\+ROOT/src/sklib/skgetv.F). This works for both zbs and ROOT, provided \textquotesingle{}SK\+\_\+\+FILE\+\_\+\+FORMAT\textquotesingle{} is set first (0=zbs, 1=root). For zbs\+: populates the {\ttfamily skvect\+\_\+} common block with data from MCVERTEX and MCVECT banks (skdetsim primaries). n.\+b. it only returns the last vertex in MCVERTEX in skvect\+\_\+.\+pos, no others. number of initial particles is skvect\+\_\+.\+nvect (capped at 50). particle codes are returned in skvect\+\_\+.\+ip\mbox{[}0..NVECT-\/1\mbox{]}, initial positions in skvect\+\_\+.\+pin\mbox{[}0..NVECT-\/1\mbox{]}\mbox{[}0..2\mbox{]}, and initial momentum in skvect\+\_\+.\+pabs\mbox{[}0..NVECT-\/1\mbox{]}.

populates the {\ttfamily skvect\+\_\+add\+\_\+} common block with additional data from the MCPARMCONV bank (e.\+g. dark rate, random seeds, trigger config etc. -\/ i.\+e. DARKDS, TCNVSK, QCNVSK, DTHRSK etc.) For root\+: uses fortran interface skroot\+\_\+get\+\_\+mc(). Only last primary vertex and primary particles populated.

secondaries are retreieved by either\+: for root\+: copying data from the SECONDARY branch in particular\+: iprtscnd (pdg), tscnd (time), vtxscnd (vertex), pscnd (momentum), lmecscnd (creation process), iprntprt (parent pdg) branches are used to construct \mbox{\hyperlink{classParticle}{Particle}} class instances N.\+B. only some secondaries are copied\+: neutrons and {\itshape some hydrogen capture products}\+: -\/$>$ deuterons, gammas, and electrons over ckv threshold \char`\"{}from an interaction other than multiple scattering\char`\"{} (lmecscnd!=2) Note also\+: particles where the secondary creation vertex was outside the ID (radius $>$ RINTK, height $>$ ZPINTK) or where it was within a PMT (determined via fortran function inpmt\+\_\+()) are skipped. electrons and deuterons are not associated to their respective capture, only gammas are -\/ by their creation time being within 1e-\/7 ns of an existing capture....! for zbs\+: apflscndprt\+\_\+(); // note the following are not all the same. Not sure which is being used! \$\+ATMPD\+\_\+\+ROOT/src/analysis/ndecay/analyses/p2muk0/official\+\_\+ntuple\+\_\+pdkfit/apflscndprt.F \$\+ATMPD\+\_\+\+ROOT/src/analysis/neutron/ntag/apflscndprt.F \$\+ATMPD\+\_\+\+ROOT/src/analysis/neutron/ntag\+\_\+gd/apflscndprt.F \$\+ATMPD\+\_\+\+ROOT/src/analysis/official\+\_\+ntuple/apflscndprt.F \$\+ATMPD\+\_\+\+ROOT/src/recon/fitqun/apflscndprt.F

based on usage this populates secndprt\+\_\+ with the same members are detailed below (and possibly more...)

Note\+: the \textquotesingle{}variables\textquotesingle{} tree \textquotesingle{}true\+\_\+neutron\+\_\+count\textquotesingle{} branch is set as the number of neutron captures, as identified by the secondaries scan. A neutron capture is identified by finding a secondary product created by process lmec=18, at a unique time. 